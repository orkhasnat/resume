%% If you need to pass whatever options to xcolor
\PassOptionsToPackage{dvipsnames}{xcolor}

%% If you are using \orcid or academicons
%% icons, make sure you have the academicons
%% option here, and compile with XeLaTeX
%% or LuaLaTeX.
% \documentclass[10pt,a4paper,academicons]{altacv}

%% Use the "normalphoto" option if you want a normal photo instead of cropped to a circle
% \documentclass[10pt,a4paper,normalphoto]{altacv}

%% Fork: CV dark mode toggle enabler to use a inverted color palette.
%% Use the "darkmode" option if you want a color palette used to 
% \documentclass[10pt,a4paper,darkmode]{altacv}

\documentclass[10pt,a4paper,ragged2e,withhyper]{altacv}

%% AltaCV uses the fontawesome5 and academicons fonts
%% and packages.
%% See http://texdoc.net/pkg/fontawesome5 and http://texdoc.net/pkg/academicons for full list of symbols. You MUST compile with XeLaTeX or LuaLaTeX if you want to use academicons.

% Change the page layout if you need to
\geometry{left=1.2cm,right=1.2cm,top=1cm,bottom=1cm,columnsep=0.7cm}

% The paracol package lets you typeset columns of text in parallel
\usepackage{paracol}
\usepackage[super]{nth}


% Change the font if you want to, depending on whether
% you're using pdflatex or xelatex/lualatex
\ifxetexorluatex
  % If using xelatex or lualatex:
  \setmainfont{Roboto Slab}
  \setsansfont{Lato}
  \renewcommand{\familydefault}{\sfdefault}
\else
  % If using pdflatex:
  \usepackage[rm]{roboto}
  \usepackage[defaultsans]{lato}
  % \usepackage{sourcesanspro}
  \renewcommand{\familydefault}{\sfdefault}
\fi

% Fork: Change the color codes to test your personal variant on any mode
\ifdarkmode%
  \definecolor{PrimaryColor}{HTML}{0F52D9}
  \definecolor{SecondaryColor}{HTML}{3F7FFF}
  \definecolor{ThirdColor}{HTML}{F3890B}
  \definecolor{BodyColor}{HTML}{ABABAB}
  \definecolor{EmphasisColor}{HTML}{ABA2A2}
  \definecolor{BackgroundColor}{HTML}{242424}
\else%
  % \definecolor{PrimaryColor}{HTML}{001F5A}
  \definecolor{PrimaryColor}{HTML}{290001}
  % \definecolor{SecondaryColor}{HTML}{0039AC}
  \definecolor{SecondaryColor}{HTML}{87431D}%{FF4949}
  % \definecolor{ThirdColor}{HTML}{FF6000}
  \definecolor{ThirdColor}{HTML}{1C0A00}
  % \definecolor{BodyColor}{HTML}{666666}
  \definecolor{BodyColor}{HTML}{1a0e00}
  \definecolor{EmphasisColor}{HTML}{2E2E2E}
  \definecolor{BackgroundColor}{HTML}{E2E2E2}
\fi%

\colorlet{name}{PrimaryColor}
\colorlet{tagline}{PrimaryColor}
\colorlet{heading}{PrimaryColor}
\colorlet{headingrule}{ThirdColor}
\colorlet{subheading}{SecondaryColor}
\colorlet{accent}{SecondaryColor}
\colorlet{emphasis}{EmphasisColor}
\colorlet{body}{BodyColor}
\pagecolor{white}

% Change some fonts, if necessary
\renewcommand{\namefont}{\Huge\rmfamily\bfseries}
\renewcommand{\personalinfofont}{\small\bfseries}
\renewcommand{\cvsectionfont}{\LARGE\rmfamily\bfseries}
\renewcommand{\cvsubsectionfont}{\large\bfseries}

% Change the bullets for itemize and rating marker
% for \cvskill if you want to
\renewcommand{\itemmarker}{{\small\textbullet}}
\renewcommand{\ratingmarker}{\faCircle}

%% sample.bib contains your publications
%% \addbibresource{sample.bib}

\begin{document}
    \name{Tasnimul Hasnat}
    \tagline{\small BSc. in Computer Science and Engineering | Cybersecurity Analyst | Software Developer | Open Source Enthusiast }
    
    \personalinfo{
        \email{orkhasnat@gmail.com}
        \phone{+880 1731969827}
        \github{orkhasnat}{https://github.com/orkhasnat}
        \linkedin{Tasnimul Hasnat}{https://www.linkedin.com/in/tasnimul-hasnat-37515025a/}
        % \location{Dhaka, Bangladesh}
        %\homepage{nicolasomar.me}
        %\medium{nicolasomar}
        %% You MUST add the academicons option to \documentclass, then compile with LuaLaTeX or XeLaTeX, if you want to use \orcid or other academicons commands.
        % \orcid{0000-0000-0000-0000}
        %% You can add your own arbtrary detail with
        %% \printinfo{symbol}{detail}[optional hyperlink prefix]
        % \printinfo{\faPaw}{Hey ho!}[https://example.com/]
        %% Or you can declare your own field with
        %% \NewInfoFiled{fieldname}{symbol}[optional hyperlink prefix] and use it:
        % \NewInfoField{gitlab}{\faGitlab}[https://gitlab.com/]
        % \gitlab{your_id}
    }
    
    \makecvheader
    %% Depending on your tastes, you may want to make fonts of itemize environments slightly smaller
    % \AtBeginEnvironment{itemize}{\small}
    
    %% Set the left/right column width ratio to 6:4.
    \columnratio{0.35}

    % Start a 2-column paracol. Both the left and right columns will automatically
    % break across pages if things get too long.
    \begin{paracol}{2}
        % ----- STRENGTHS -----
        \cvsection{Interests}
            \cvtag{Linux}
            \cvtag{Networking}
            \cvtag{History}\\
            \cvtag{Digital Forensics}
            \cvtag{OSINT}\\
            \cvtag{Economics}
            \cvtag{Reverse Engineering}\\
        \cvsection{Skills}
        \cvsubsection{Programming Languages}
            \cvtag{C}
            \cvtag{C++}
            \cvtag{Javascript}
            \cvtag{Bash}\\
            \cvtag{Python}
            \cvtag{Java}
            \cvtag{x86 Assembly}\\
        \cvsubsection{Frameworks \& Libraries}
            \cvtag{SFML}
            \cvtag{JavaFX}
            \cvtag{Angr}\\
            \cvtag{EJS}
            \cvtag{Tailwind}
            \cvtag{React}\\
            \cvtag{Astro}
            \cvtag{ExpressJS}\\
        \cvsubsection{Tools}
            \cvtag{Git}
            \cvtag{Blender}
            \cvtag{GDB}  
            \cvtag{Autopsy}
            \cvtag{Docker}
            \cvtag{GIMP}
            \cvtag{Wireshark}\\
            \cvtag{Ghidra}
            \cvtag{Burp Suite}\\
        % \cvsubsection{Styling Languages}
        %     \cvtag{Sass}
        %     \cvtag{Markdown}
        %     \cvtag{Latex}
        %     \cvtag{YAML}\\
    
        \cvsubsection{Database}
            \cvtag{Oracle}
            \cvtag{MySQL}
            \cvtag{Mariadb}\\
            
        % \cvsection{Experience}
        %  \begin{itemize}[label = \color{accent}\faCube]
        %     \item \color{BodyColor} Took part in Cybersecurity Internship program 

        % \end{itemize} 
       
        
         \cvsection{Co-curriculars}
         \begin{itemize}[label = \color{accent}\faCube]
            \item \color{BodyColor} Organizer of Intra IUT Coderush CTF Competition.
            \item \color{BodyColor} Problem Setter at KnightCTF 2023.
            % \item \color{BodyColor} Problem Setter at Flaghunt 2023.
            \item \color{BodyColor} Volunteer at Intra IUT Hackathon Competition.
            \item \color{BodyColor} Instructor at IUT CTF Club.
            \item \color{BodyColor} Participated in various national and international CTF (Capture the Flag) competitions since 2022.

        \end{itemize}    

        \newpage
        
        %% Switch to the right column. This will now automatically move to the second
        %% page if the content is too long.
        \switchcolumn
        
        % ----- EDUCATION -----
        \cvsection{Education}
             \edu{\color{BodyColor}Islamic University of Technology}{B.Sc. in Computer Science and Engineering}{Gazipur, Bangladesh}{Jan 2020 -- Present}{CGPA: 3.77 (Up to \nth{7} Semester)}
             \edu{\color{BodyColor}Dhaka Residential Model College}{Higher Secondary Certificate}{Dhaka, Bangladesh}{2017 -- 2019}{GPA: 5.00}
        % ----- EDUCATION -----
        

        \cvsection{ACHIEVEMENTS}
        
            \begin{itemize}[label = \color{accent}\faTrophy] 
                \item {\color{BodyColor}\textbf{BUET CSE Fest CTF 2023}}\\
                \textbf{Champion} among 100+ teams
                \vspace{0.5em}
                \item {\color{BodyColor}\textbf{DU Cefalo ITVerse CTF 2023 }}\\
                \textbf{\small \nth{4}} among 60+ teams
                \vspace{0.5em}
                \item {\color{BodyColor}\textbf{RITSEC International CTF 2023}}\\\textbf{\small \nth{7}}  among 710+ \textit{international} teams
                \vspace{0.5em}
                \item {\color{BodyColor}\textbf{SUST SWE Technovent  CTF 2023}}\\\textbf{\small \nth{6}} among 40+ teams
                \vspace{0.5em}
                \item {\color{BodyColor}\textbf{RIOT Flaghunt CTF 2022}}\\\small \textbf{\nth{8}} among 80+ teams
                \vspace{.05em}
                \item {\color{BodyColor}\textbf{Awarded IUT-OIC Partial Scholarship}}\\\small Ranked \nth{195} amongst 4200+ participants. Scholarship awarded for 3 Years equivalent to \textbf{\$13500}.
            \end{itemize}
        
        % ----- PROJECTS -----
        \cvsection{Projects}
        
            \cvproject{\color{BodyColor}Abaash}{\cvrepo{| \faGithub}{https://github.com/orkhasnat/Abaash}}{Nov 2022 -- Dec 2022}{ EJS, ExpressJs, Mariadb, Bulma}
             Abaash is a web-based platform that simplifies the process of finding and renting flats for non-residential IUT students. Abaash is the successor to Sanctuary.
            
            \cvproject{\color{BodyColor}Fox's Tale}{\cvrepo{| \faGithub}{https://github.com/orkhasnat/Foxs-Tale}}{Oct 2021 -- Nov 2021}{ C++, SFML}
            Fox's Tale is a tribute to the classic Rapid Roll game. It adds new features such as updated graphics, sound effects, obstacles, power-ups, and achievements to the original game. Nostalgic fan or a new player, it provides an engaging gaming experience that tests your reflexes, agility, and strategic thinking.
\\
            \cvproject{\color{BodyColor}Sanctuary}{\cvrepo{| \faGithub}{https://github.com/orkhasnat/Sanctuary}}{Jun 2022 -- Aug 2022}{ Java, JavaFX, FXML, SQL}
             Sanctuary is a user-friendly desktop application, offering non-residential IUT students the ability to find flats based on their preferences, and allows flat owners to easily manage their properties and tenants.

            \cvproject{\color{BodyColor}Outbreak}{\cvrepo{| \faGithub}{https://github.com/orkhasnat/Outbreak}}{Jan 2021 -- March 2021}{ C++, SFML}
             Outbreak is a viral simulator. It models the spread of viruses based on parameters such as transmission rate, mortality rate, and incubation period. Users can manipulate these parameters to see how they affect the spread of the virus.
             
        % 
        
    \end{paracol}
\end{document}
